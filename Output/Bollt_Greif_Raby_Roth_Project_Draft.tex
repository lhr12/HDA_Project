\documentclass[12pt,]{article}
\usepackage{lmodern}
\usepackage{amssymb,amsmath}
\usepackage{ifxetex,ifluatex}
\usepackage{fixltx2e} % provides \textsubscript
\ifnum 0\ifxetex 1\fi\ifluatex 1\fi=0 % if pdftex
  \usepackage[T1]{fontenc}
  \usepackage[utf8]{inputenc}
\else % if luatex or xelatex
  \ifxetex
    \usepackage{mathspec}
  \else
    \usepackage{fontspec}
  \fi
  \defaultfontfeatures{Ligatures=TeX,Scale=MatchLowercase}
    \setmainfont[]{Times New Roman}
\fi
% use upquote if available, for straight quotes in verbatim environments
\IfFileExists{upquote.sty}{\usepackage{upquote}}{}
% use microtype if available
\IfFileExists{microtype.sty}{%
\usepackage{microtype}
\UseMicrotypeSet[protrusion]{basicmath} % disable protrusion for tt fonts
}{}
\usepackage[margin=2.54cm]{geometry}
\usepackage{hyperref}
\hypersetup{unicode=true,
            pdftitle={Impacts of Land Use on Water Quality in Minnesota},
            pdfauthor={Keith Bollt, Jake Greif, Felipe Raby-Amadori, \& Lindsay Roth},
            pdfborder={0 0 0},
            breaklinks=true}
\urlstyle{same}  % don't use monospace font for urls
\usepackage{color}
\usepackage{fancyvrb}
\newcommand{\VerbBar}{|}
\newcommand{\VERB}{\Verb[commandchars=\\\{\}]}
\DefineVerbatimEnvironment{Highlighting}{Verbatim}{commandchars=\\\{\}}
% Add ',fontsize=\small' for more characters per line
\usepackage{framed}
\definecolor{shadecolor}{RGB}{248,248,248}
\newenvironment{Shaded}{\begin{snugshade}}{\end{snugshade}}
\newcommand{\AlertTok}[1]{\textcolor[rgb]{0.94,0.16,0.16}{#1}}
\newcommand{\AnnotationTok}[1]{\textcolor[rgb]{0.56,0.35,0.01}{\textbf{\textit{#1}}}}
\newcommand{\AttributeTok}[1]{\textcolor[rgb]{0.77,0.63,0.00}{#1}}
\newcommand{\BaseNTok}[1]{\textcolor[rgb]{0.00,0.00,0.81}{#1}}
\newcommand{\BuiltInTok}[1]{#1}
\newcommand{\CharTok}[1]{\textcolor[rgb]{0.31,0.60,0.02}{#1}}
\newcommand{\CommentTok}[1]{\textcolor[rgb]{0.56,0.35,0.01}{\textit{#1}}}
\newcommand{\CommentVarTok}[1]{\textcolor[rgb]{0.56,0.35,0.01}{\textbf{\textit{#1}}}}
\newcommand{\ConstantTok}[1]{\textcolor[rgb]{0.00,0.00,0.00}{#1}}
\newcommand{\ControlFlowTok}[1]{\textcolor[rgb]{0.13,0.29,0.53}{\textbf{#1}}}
\newcommand{\DataTypeTok}[1]{\textcolor[rgb]{0.13,0.29,0.53}{#1}}
\newcommand{\DecValTok}[1]{\textcolor[rgb]{0.00,0.00,0.81}{#1}}
\newcommand{\DocumentationTok}[1]{\textcolor[rgb]{0.56,0.35,0.01}{\textbf{\textit{#1}}}}
\newcommand{\ErrorTok}[1]{\textcolor[rgb]{0.64,0.00,0.00}{\textbf{#1}}}
\newcommand{\ExtensionTok}[1]{#1}
\newcommand{\FloatTok}[1]{\textcolor[rgb]{0.00,0.00,0.81}{#1}}
\newcommand{\FunctionTok}[1]{\textcolor[rgb]{0.00,0.00,0.00}{#1}}
\newcommand{\ImportTok}[1]{#1}
\newcommand{\InformationTok}[1]{\textcolor[rgb]{0.56,0.35,0.01}{\textbf{\textit{#1}}}}
\newcommand{\KeywordTok}[1]{\textcolor[rgb]{0.13,0.29,0.53}{\textbf{#1}}}
\newcommand{\NormalTok}[1]{#1}
\newcommand{\OperatorTok}[1]{\textcolor[rgb]{0.81,0.36,0.00}{\textbf{#1}}}
\newcommand{\OtherTok}[1]{\textcolor[rgb]{0.56,0.35,0.01}{#1}}
\newcommand{\PreprocessorTok}[1]{\textcolor[rgb]{0.56,0.35,0.01}{\textit{#1}}}
\newcommand{\RegionMarkerTok}[1]{#1}
\newcommand{\SpecialCharTok}[1]{\textcolor[rgb]{0.00,0.00,0.00}{#1}}
\newcommand{\SpecialStringTok}[1]{\textcolor[rgb]{0.31,0.60,0.02}{#1}}
\newcommand{\StringTok}[1]{\textcolor[rgb]{0.31,0.60,0.02}{#1}}
\newcommand{\VariableTok}[1]{\textcolor[rgb]{0.00,0.00,0.00}{#1}}
\newcommand{\VerbatimStringTok}[1]{\textcolor[rgb]{0.31,0.60,0.02}{#1}}
\newcommand{\WarningTok}[1]{\textcolor[rgb]{0.56,0.35,0.01}{\textbf{\textit{#1}}}}
\usepackage{longtable,booktabs}
\usepackage{graphicx,grffile}
\makeatletter
\def\maxwidth{\ifdim\Gin@nat@width>\linewidth\linewidth\else\Gin@nat@width\fi}
\def\maxheight{\ifdim\Gin@nat@height>\textheight\textheight\else\Gin@nat@height\fi}
\makeatother
% Scale images if necessary, so that they will not overflow the page
% margins by default, and it is still possible to overwrite the defaults
% using explicit options in \includegraphics[width, height, ...]{}
\setkeys{Gin}{width=\maxwidth,height=\maxheight,keepaspectratio}
\IfFileExists{parskip.sty}{%
\usepackage{parskip}
}{% else
\setlength{\parindent}{0pt}
\setlength{\parskip}{6pt plus 2pt minus 1pt}
}
\setlength{\emergencystretch}{3em}  % prevent overfull lines
\providecommand{\tightlist}{%
  \setlength{\itemsep}{0pt}\setlength{\parskip}{0pt}}
\setcounter{secnumdepth}{5}
% Redefines (sub)paragraphs to behave more like sections
\ifx\paragraph\undefined\else
\let\oldparagraph\paragraph
\renewcommand{\paragraph}[1]{\oldparagraph{#1}\mbox{}}
\fi
\ifx\subparagraph\undefined\else
\let\oldsubparagraph\subparagraph
\renewcommand{\subparagraph}[1]{\oldsubparagraph{#1}\mbox{}}
\fi

%%% Use protect on footnotes to avoid problems with footnotes in titles
\let\rmarkdownfootnote\footnote%
\def\footnote{\protect\rmarkdownfootnote}

%%% Change title format to be more compact
\usepackage{titling}

% Create subtitle command for use in maketitle
\providecommand{\subtitle}[1]{
  \posttitle{
    \begin{center}\large#1\end{center}
    }
}

\setlength{\droptitle}{-2em}

  \title{Impacts of Land Use on Water Quality in Minnesota}
    \pretitle{\vspace{\droptitle}\centering\huge}
  \posttitle{\par}
  \subtitle{\url{https://github.com/lhr12/HDA_Project}}
  \author{Keith Bollt, Jake Greif, Felipe Raby-Amadori, \& Lindsay Roth}
    \preauthor{\centering\large\emph}
  \postauthor{\par}
    \date{}
    \predate{}\postdate{}
  

\begin{document}
\maketitle
\begin{abstract}
Abstract tbd
\end{abstract}

\textless{}Information in these brackets are used for annotating the
RMarkdown file. They will not appear in the final version of the PDF
document\textgreater{}

\newpage
\tableofcontents 
\newpage
\listoftables 
\newpage
\listoffigures 
\newpage

\textless{}Note: set up autoreferencing for figures and tables in your
document\textgreater{}

\hypertarget{research-question-and-rationale}{%
\section{Research Question and
Rationale}\label{research-question-and-rationale}}

\begin{itemize}
\tightlist
\item
  Land use has a large impact on nutrient runoff into streams, lakes,
  and other water bodies
\item
  Minnesota has wide variety of land uses. Inlcudes large urban centers,
  natural lands, and agricultural area.
\item
  Nutrient management has been a challenge for states in the effort to
  control harmful algal blooms and coastal dead zones.
\item
  Understanding the causes of nutrient problems will better inform
  management strategies.
\end{itemize}

Research questions:

\begin{enumerate}
\def\labelenumi{\arabic{enumi}.}
\item
  What are the predictors of nutrients based on land use in watersheds
  in the state of Minnesota?
\item
  How do you characterize seasonal variation between the predictors of
  nutrients?
\end{enumerate}

Goals: * Determine how land use, watershed size, and ecoregion explain
variation in nutrient loading indicators. * Discern whether there a
seasonal trends in nutrient loading indicators based on land use types,
watershed size, and ecoregion. * Provide insight to inform decisions
about nutrient managment practices based on land use types, watershed
size, and ecoregion.

\newpage

\hypertarget{dataset-information}{%
\section{Dataset Information}\label{dataset-information}}

The data used in this analysis include data from the Lake Multi-Scaled
Geospatial and Temporal Database (LAGOSNE) and the EPA ecoregion spatial
datasets.

LAGOSNE is a collection of several data modules that contain information
on lakes in the northern United States. The modules contain data from
thousands of lakes in 17 states in the northeastern and midwestern
United States, from Missouri to Maine. The dataset includes a complete
list of all lakes bigger than 4 hectacres in the 17 state area, and
water quality data on a large number of lakes, spanning every state.

Ecoregions are used by planning managers to understand the type of land
use that occurs in different regions of the United States. There are
different levels of ecoregions. Level 1 divides North America into 15
ecological regions, while Level IV offers fine ecological resolution for
each state. This data was published by the U.S. EPA Office of Research
and Development (ORD) - National Health and Environmental Effects
Research Laboratory (NHEERL). For the purposes of our project, we
selected Level III ecoregions because they appear to offer a descriptive
narrative of the land use patterns of Minnesota without making a
`distinction without a difference'.

\begin{Shaded}
\begin{Highlighting}[]
\NormalTok{Variables.table <-}\StringTok{ }\KeywordTok{read.csv}\NormalTok{(}\StringTok{"../Output/Variable_Descriptions_Table.csv"}\NormalTok{)}
\KeywordTok{library}\NormalTok{(knitr)}
\KeywordTok{kable}\NormalTok{(Variables.table)}
\end{Highlighting}
\end{Shaded}

\begin{longtable}[]{@{}llll@{}}
\toprule
Column.Name & Description & Units & Variable.Type\tabularnewline
\midrule
\endhead
chla & Chlorophyll a & mg/L & Depedent\tabularnewline
secchi & Secchi depth & m & Depedent\tabularnewline
Urban.pct & Percent urban land cover & \% & Independent-
fixed\tabularnewline
Undeveloped.pct & Percent natural land cover & \% & Independent-
fixed\tabularnewline
Ag.pct & Percent agricultural land cover & \% & Independent-
fixed\tabularnewline
LakeIWS.Ratio & Lake surface area to watershed area ratio & N/A &
Independent- fixed\tabularnewline
Season & Early, prime, and late growing ``seasons'' & N/A & Independent-
fixed\tabularnewline
US\_L3NAME & Level 3 ecoregions & N/A & Independent-
random\tabularnewline
\bottomrule
\end{longtable}

\newpage

\hypertarget{exploratory-data-analysis-and-wrangling}{%
\section{Exploratory Data Analysis and
Wrangling}\label{exploratory-data-analysis-and-wrangling}}

\newpage

\hypertarget{analysis}{%
\section{Analysis}\label{analysis}}

\begin{itemize}
\tightlist
\item
  First we will create correlation plots in order to eliminate variables
  with a correlation greater than 0.8.
\item
  Then we will run Shapiro-Wilkes tests to determine normality and the
  need for possible data transformations.
\item
  After determining the distributions of the data, then we will generate
  mixed effect linear models with chlorophyll a and secchi depth as
  response variables, land use and watershed size as fixed effects, and
  ecoregion as a random effect.
\end{itemize}

Final figures will include: * 6 maps of the state, each showing the
relationship between land use and both response variables. Ecoregion
will be included as a base layer for each map. * Scatter plots showing
the strongest relationships between land use and the response variables.
* Table showing results of linear model.

\newpage

\hypertarget{summary-and-conclusions}{%
\section{Summary and Conclusions}\label{summary-and-conclusions}}

\begin{itemize}
\tightlist
\item
  Conclusions will include a discussion of our results within the
  context of MN nutrient management plan.
\end{itemize}


\end{document}
